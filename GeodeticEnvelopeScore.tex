\documentclass{article}
\author{David Gore}
\title{The Geodetic Envelope Score}
\date{December 19, 2025}

\usepackage[
    backend=biber,
    style=numeric,
    sorting=none
    ]{biblatex}
\addbibresource{references.bib}

\usepackage{amsthm}
\usepackage{amsmath}
\usepackage{graphicx}
\usepackage{float}

% Numbered examples without section number
\newtheorem{example}{Example}
\newtheorem*{definition}{Definition}

\begin{document}
\maketitle

\section{Introduction}
Several metrics have been developed to quantify the degree to which congressional districts have
been gerrymandered. Many of these metrics measure the compactness of a district by 
comparing a geometrical property of the district, such as area or perimeter, to a geometrical property 
in a related ideal shape. The ideal shapes are typically circles or rectangles. 
For example, the Reock score is the ratio of the area of a district to the area of the minimum 
circumscribing circle of the district. The Polsby-Popper score is the ratio of the area of a district 
to the area of a circle whose perimeter is the same as the district's perimeter. The Schwartzberg score
is the ratio of the perimeter of a district to the perimeter of a circle having the same area as the 
district.

\section{The Geodetic Envelope Score}
The book ``How The States Got Their Shapes'', \cite{stein}, documents the history of how the states got 
their individual boundaries. Boundaries can be natural, political, based on the division of resources, 
based on a principle such as equality, or be given in the King's charter. In all cases, the most common state 
delineators were defined by ``parallels'', lines of latitude, and ``meridians'', line of longitude. 
Kansas, Nebraska, South Dakota and North Dakota all have 3° of latitudinal height. While
Colorado, Wyoming and Montana all have 4°. The states Washington, Oregon, Colorado, Wyoming,
North Dakota and South Dakota all have about 7° of longitudinal width. The southern borders of Virginia,
Kentucky and Missouri are all at 37.5° latitude. Indeed a large section of the northern border of the 
United States is at 49° latitude. Thus, it seems natural to use lines of latitude and longitude in the 
definition of a compactness score for voting districts.

A \textit{geodetic envelope} or simply \textit{envelope} is a Geographic Information System (GIS) / Open Geospatial
Consortium (OGC) designated term defining a region specified by two latitude-longitude points. The two points are 
taken to be $(\lambda_{min},\varphi_{min})$ and $(\lambda_{max},\varphi_{max})$, where $\lambda$ denotes
longitude and $\varphi$ denotes geodetic latitude.

The geodetic envelope score of a district is the area of the district divided by the surface area of the 
geodetic envelope containing the district,
\begin{equation}
    GE = \frac{A(D)}{A(S)}
\end{equation}
Here, $D$ is the district and $S$ is the bounding envelope of $D$. $S$ is a surface on the reference ellipsoid, 
a surface which approximates the earth geoid.

\subsection{Envelope Area as a Surface of Revolution}
\label{sec:SREllipse}
The envelope area of $S$, is the surface area of $S$, denoted $A(S)$. It can be computed by rotating 
the ``meridian'' ellipse, ${x^2}/{a^2} + {y^2}/{b^2} = 1$ with semi-major axis $a=6378.137$ km and 
semi-minor axis $b=6356.752$ km about its minor axis. 
The values for $a$ and $b$ are defined by the world geodetic system (WGS)-84. 
Also, the first eccentricity of the earth, $e$ is related to the ellipses' axes by the formula
$e=((a^2-b^2)/a^2)^{1/2}$ and has the value $e=0.081819$, \cite{noureldin}. 
The area of the surface generated by rotating a horizontal
element of area from $y=y_1$ to $y=y_2$ about the y-axis is
\begin{equation*}
    A(S) = 2\pi(\frac{a}{b})\int_{y_1}^{y_2}\sqrt{b^2+e'^2y^2} dy
\end{equation*}
In this equation, $e'$ is called the second eccentricity of the earth. 
Its formula is $e'=((a^2-b^2)/b^2)^{1/2}$ and it has the value $e'=0.082094$. 
To check the validity of this equation, set $a=b$, $e'=0$ and $y_1=0$, $y_2=a$, 
then the area evaluates to $A(S)=2\pi a^2$, the surface area of a hemi-sphere. 
Evaluating the integral in leads to a formula for the surface area of an envelope
\begin{equation}
    A(S) = \frac{ab}{2e'}(sec(u)tan(u)+ln|sec(u)+tan(u)|)\Big|_{u_1}^{u_2}(\lambda_2-\lambda_1)(\frac{\pi}{180})
\label{eq:SAEllipsoid}
\end{equation}
Here, longitude ($\lambda$) and latitude ($\varphi$) are given in degrees.
The limits of integration are $u_1 = tan^{-1}((e'/b)y_1)$, $u_2 = tan^{-1}((e'/b)y_2)$ where 
$y_1 = N(\varphi_1)(1-e^2)sin(\varphi_1)$ and $y_2 = N(\varphi_2)(1-e^2)sin(\varphi_2)$. 
A derivation is given in the Appendix.

\subsection{Envelope Area by Map Projection}
Another possibility is to project the envelope onto a planar region in two dimensional Euclidean space 
and compute the area there. The area of the planar region can be computed as a line integral over the
boundary of the region using Green's Theorem. It can be shown that this line integral is equal to,
\begin{equation*}
    A = \frac{1}{2} \oint x \,dy - y \,dx
\end{equation*}
Reference \cite{engmath} points out the interesting fact that this theorem is the principle behind 
how certain planimeters (devices which measure area) work.

Suppose a geodetic envelope is defined by the WGS-84 coordinates $(\lambda_1,\varphi_1)$ and 
$(\lambda_2,\varphi_2)$. That is, the envelope consists of the vertices of the polygon
$A=(\lambda_{min},\varphi_{min})$, $B=(\lambda_{max},\varphi_{min})$, $C=(\lambda_{max},\varphi_{max})$
and $D=(\lambda_{min},\varphi_{max})$.
The Equal Earth Projection \cite{ee2018} is a modern equal-area, pseudocylindrical map projection. It shall
accomplish the task of projecting the envelope vertices onto a planar region. Since the vertices $A$ and
$B$ have the same latitude, they will be projected onto points with the same $y$ value. Thus, denote the 
projected points $A'=(x_1,y_1)$, $B'=(x_2,y_1)$, $C'=(x_3,y_2)$ and $D'=(x_4,y_2)$. Let $C$ denote the
boundary of the projected envelope. Then a parameterization for $C$ in the positive counter-clockwise
direction is,
\[
\begin{cases}
    \begin{array}{lll}
        x(t) = x_1 + (x_2-x_1)t, & y=y_1; & 0 \le t < 1, \\
        x(t) = x_2 + (x_3-x_2)(t-1), & y=y_1 + (y_2-y_1)(t-1); & 1 \le t < 2, \\
        x(t) = x_3 + (x_4-x_3)(t-2), & y=y_2; & 2 \le t < 3, \\
        x(t) = x_4 + (x_1-x_4)(t-3), & y=y_2 + (y_1-y_2)(t-3); & 3 \le t \le 4.        
    \end{array}
\end{cases}
\]
Using this parameterization to evaluate the line integral stipulated by Green's Theorem gives,
\begin{equation}
    \begin{split}
        A(S) &= \frac{1}{2}(-y_1(x_2-x_1) + x_2(y_2-y_1) \\
        &\quad - y_1(x_3-x_2) - y_2(x_4-x_3) + x_4(y_1-y_2) - y_2(x_1-x_4))        
    \end{split}
\label{eq:SAProjection}
\end{equation}

This formula to compute the surface area of the envelope is an alternative to \eqref{eq:SAEllipsoid}. 
The relative 
differences in percent between using \eqref{eq:SAEllipsoid} and \eqref{eq:SAProjection} 
have a maximum value of less than 0.1\%.

\subsection{Envelope Area using Spherical Earth Model}
If, instead of the reference ellipsoid, a sphere of radius $a$ is taken as an approximation to the earth geoid,
\eqref{eq:SAEllipsoid} simplifies to 
\begin{equation}
    A(S) = a^2(sin(\varphi_2)-sin(\varphi_1))(\lambda_2-\lambda_1)(\frac{\pi}{180})
\label{eq:SASphere}
\end{equation}
As above, longitude ($\lambda$) and latitude ($\varphi$) are given in degrees.
This formula can be derived in a manner similar to what is shown in the Appendix for the derivation
of \eqref{eq:SAEllipsoid}.
The differences in calculated area of envelopes between using \eqref{eq:SAEllipsoid} and 
\eqref{eq:SASphere} are shown in the following figure. In this figure the
districts have been sorted by the median latitude of the envelope. Hence, districts with lower
district numbers have a smaller median latitude than districts with higher district numbers.
\begin{figure}[H]
    \centering
    \includegraphics[width=0.7\textwidth]{cmpAreaSphere.png}
    \caption{Envelope Area Differences with Spherical Earth-Model}
    \label{fig:CmpAreaSphere}
\end{figure}

The minimum value of -588 $km^2$ at index $23$ is district TX23. This Texas district is on the 
United States' southern border with Mexico.
The maximum value of 657 $km^2$ at index $415$ is district MT02. This Montana district is on the 
United States' northern border with Canada. In other words, the spherical approximation overestimates
the envelope area for states in the southern United States and underestimates envelope area for states 
in the northern United States.

The relative error in percent of the spherical earth model approximation ranges from a minimum value
of 0.0009\% at latitude 37.785\textdegree to maximum values 0.2634\%, 0.2454\% at latitudes 
25.101\textdegree, 48.390\textdegree, respectively.

\subsection{Envelope Area using Spherical Geometry}
An envelope can be represented geometrically as the difference between two spherical triangles. 
Again, suppose the envelope is defined by the WGS-84 coordinates $(\lambda_1,\varphi_1)$ and 
$(\lambda_2,\varphi_2)$. The larger triangle consists of the vertices $(\lambda_1,\varphi_{min})$,
$(\lambda_2,\varphi_{min})$ and the North Pole and the smaller triangle consists of the vertices 
$(\lambda_1,\varphi_{max})$, $(\lambda_2,\varphi_{max})$ and the North Pole. 
Since these triangles are defined by WGS-84 coordinates, they shall be referred to as
navigational spherical triangles. In any case, the area of the 
envelope is the area of the larger triangle minus the area of the smaller triangle.

The area of a spherical triangle can be computed using Gerard's theorem, \cite{strig}.
Let $E$ denote the spherical excess of the triangle, then the area of a spherical triangle on a
sphere of radius $R$ is
\begin{equation}
    A = (E-\pi)R^2
\label{eq:SAGirard}
\end{equation}
The spherical excess is the sum of the angles of the spherical triangle and 
in this formula it is specified in radians.

Therefore, to determine the area of a navigational spherical triangle, the interior angles need to be
found. In the field of Navigation, this is a known as the Inverse Problem of Geodesy \cite{volpe}
\begin{definition}[Inverse Problem of Geodesy]
Given points $A$ and $B$ with coordinates
$(\lambda_A,\varphi_A)$ and $(\lambda_B,\varphi_B)$, respectively, find the geocentric angle $\theta$
connecting $A$ and $B$ and the azimuth angles $\psi_{B/A}$ and $\psi_{A/B}$ from north 
of the path at each end.    
\end{definition}
The azimuth angles can be computed with the formulas,
\begin{equation*}
    tan(\psi_{B/A})=\frac{cos(\varphi_B)sin(\lambda_B-\lambda_A)}{sin(\varphi_B)cos(\varphi_A)-cos(\varphi_B)sin(\varphi_A)cos(\lambda_B-\lambda_A)}
\end{equation*}
and
\begin{equation*}
    tan(\psi_{A/B})=\frac{cos(\varphi_A)sin(\lambda_A-\lambda_B)}{sin(\varphi_A)cos(\varphi_B)-cos(\varphi_A)sin(\varphi_B)cos(\lambda_A-\lambda_B)}
\end{equation*}
The third angle of the navigational triangle is the angle at the North Pole. The value of this angle is equal to 
$\lambda_B - \lambda_A$. The spherical excess $E$ is the sum of these three angles: the azimuth angles 
and the angle at the North Pole. 

All together four methods have been presented to calculate envelope area.
Taking the surface of revolution method discussed in Section~\ref{sec:SREllipse} as the benchmark,
the following figure summarizes the differences in calculating the envelope area. 
This plot shows the distribution of differences of area from the benchmark
when using equations \eqref{eq:SAProjection}, \eqref{eq:SASphere}, and \eqref{eq:SAGirard}.

\begin{figure}[H]
    \centering
    \includegraphics[width=0.7\textwidth]{cmpAreaCDF.png}
    \caption{Comparison of Envelope Area Differences}
    \label{fig:CmpAreaCDF}
\end{figure}
The Projection method is the closest in agreement with the benchmark since both methods assume an
ellipsoidal earth-model. There are minor deviations in area for larger size envelopes.
Both the spherical surface of revolution formula \eqref{eq:SASphere} and spherical geometrical
formula \eqref{eq:SAGirard} are in good agreement with the benchmark. The deviations in these
two methods correspond to smaller and larger latitudes, as shown in Figure~\ref{fig:CmpAreaSphere}. 
As expected, both spherical earth-models are in good agreement with each other.

\section{Comparison with Polsby-Popper Score}
As representative to compare with the Polsby-Popper score was chosen. 
The Polsby-Popper (PP) score \cite{pp1991} is the ratio of the area of a district to the area of a circle 
whose circumference is equal to the perimeter of the district. If $A$ is the area operator and $P$ 
the perimeter operator then the PP-score is

\begin{equation*}
PP = \frac{A(D)}{A(R)}
\end{equation*}
Here, $D$ represents a district and $R$ a circle with $P(D) = P(R)$. If $d$ is the diameter of circle $R$ then 
$A(R)=\pi(d^2/2)$ and by substitution,

\begin{equation}
PP = 4\pi\frac{A(D)}{P(D)^2}
\label{eq:PP}
\end{equation}
Two applications of \eqref{eq:PP} are shown in the following examples.
\begin{example}
Suppose $D$ is a rectangle with height $h$ and length $l$. The Polsby-Popper score for the rectangle is
\[
PP = \pi\frac{lh}{(l+h)^2}
\]
If $l=h$ (a square) then $PP=0.79$, if $l=2$, $h=1$ then $PP=0.70$, and if $l=4$, $h=1$ then $PP=0.50$.
\end{example}

\begin{example}
Suppose $D$ is an ellipse with semi-major axis $a$ and semi-minor axis $b$. Let $C$ denote the boundary of
the ellipse and choose the parametric equations for $C$ to be $x=acos(t)$, $y=bsin(t)$ for $0 \le t \le 2\pi$. 
Thus $C$ is traversed in the counter-clockwise (positive) direction. Then using Green's Theorem in the 
plane \cite{engmath}, with $P=-y/2$ and $Q=x/2$,
\[
A(D) = \oint_C -\frac{y}{2} \, dx + \frac{x}{2} \, dy = \int_0^{2\pi} [\frac{ab}{2}sin^2(t)+\frac{ab}{2}cos^2(t)] \, dt = \pi{ab}
\]
Furthermore, the perimeter of the ellipse can be written \cite{abramowitz}
\[
P(D) = 4aE(e)
\]
where $E$ is the elliptical integral of the second kind and $e$ is the eccentricity of the ellipse. The
formulas for $E$ and $e$ are shown in the Appendix. The Polsby-Popper score for the ellipse is
\[
PP = \frac{\pi}{4}\frac{b}{a}[E^2(e)]^{-1}
\]
If $a=b$ (a circle) then $PP=1$, if $a=2$, $b=1$ then $PP=0.71$, and if $a=4$, $b=1$ then $PP=0.49$.
\end{example}
These examples show that comparing the area of a district $D$ to the area of a circle $R$ 
can lead to low compactness scores even for reasonably shaped districts 
such as rectangles or ellipses. 
This fact, illustrated for the Polsby-Popper score, would also hold true for the other metrics
based on the circle: Roeck and Schwartzberg.
This is obvious since the circle doesn't cover these reasonably shaped districts very well.

\section{Results}
In the results of this section the area of the geodetic envelope was computed using \eqref{eq:SAEllipsoid}.
The following histogram compares the Geodetic Envelope with the Polsby-Popper score for all districts. 
The average value of the Geodetic Envelope score, $0.4724$, is almost twice that of the 
Polsby-Popper score, $0.2465$. This implies that geodetic envelopes are more suited as a compactness 
measure for districts than circles.
\begin{figure}[H]
    \centering
    \includegraphics[width=0.7\textwidth]{hist.png}
    \caption{Histogram of Geodetic Envelope and Polsby-Popper score.}
    \label{fig:Histogram}
\end{figure}

The following table compares the districts with the top 5 Geodetic Envelope and Polsby-Popper scores.
\begin{table}[H]
\centering
\begin{tabular}{|c|c|}
\hline
Geodetic Envelope & Polsby-Popper \\ \hline
\begin{tabular}{c|c|c}
Rank & District & Score \\
\hline
1 & WY00 & 0.9885 \\ \hline
2 & ND00 & 0.9281 \\ \hline
3 & IN07 & 0.9097 \\ \hline
4 & MT02 & 0.8826 \\ \hline
5 & NV02 & 0.8629 \\
\end{tabular}
&
\begin{tabular}{c|c|c}
Rank & District & Score \\
\hline
1 & WY00 & 0.7596 \\ \hline
2 & IN07 & 0.6953 \\ \hline
3 & NV02 & 0.5769 \\ \hline
4 & FL15 & 0.5646 \\ \hline
5 & CO05 & 0.5573 \\
\end{tabular}
\\ \hline
\end{tabular}
\caption{Top 5 Districts}
\label{tab:top5-districts}
\end{table}
District ND00 encompasses the entire state of North Dakota and when viewed on a map 
its shape is a reasonable looking
rectangle. Its Geodetic Envelope score of $0.9281$ ranks second among all districts. 
If the Polsby-Popper score is used, however, ND00 ranks 20th with a score of $0.4761$. This further supports
the notion that the Geodetic Envelope is a more suitable measure of compactness than Polsby-Popper.

The districts AK00 and HI02 in Alaska and Hawaii, respectively, have by far the worst compactness scores.
For example, the Polsby-Popper score of AK00 is $O(10^{-4})$. Thus, when comparing the bottom 5 districts,
these states were considered outliers and removed from the analysis. The next table compares the districts 
with the lowest Geodetic Envelope and Polsby-Popper scores 
\begin{table}[H]
\centering
\begin{tabular}{|c|c|}
\hline
Geodetic Envelope & Polsby-Popper \\ \hline
\begin{tabular}{c|c|c}
Rank & District & Score \\
\hline
1 & CA11 & 0.0424 \\ \hline
2 & CA42 & 0.0677 \\ \hline
3 & TX35 & 0.1137 \\ \hline
4 & FL28 & 0.1192 \\ \hline
5 & OH11 & 0.1399 \\
\end{tabular}
&
\begin{tabular}{c|c|c}
Rank & District & Score \\
\hline
1 & CA42 & 0.0260 \\ \hline
2 & CA11 & 0.0295 \\ \hline
3 & TX33 & 0.0366 \\ \hline
4 & LA06 & 0.0505 \\ \hline
5 & CA41 & 0.0596 \\
\end{tabular}
\\ \hline
\end{tabular}
\caption{Bottom 5 Districts}
\label{tab:bottom5-districts}
\end{table}
The districts CA11 and CA42 have the lowest scores for both compactness measures. 
When using the Geodetic Envelope Score, the relative difference in percent
between these two districts is $((.0677-.0424) / .05505) * 100 = 45.96\%$. For the Polsby-Popper score,
the relative difference is $12.62\%$. Therefore, the Geodetic Envelope Score does a better job of
distinguishing districts with low compactness scores.

Next, a state-wide comparison is made by calculating the average of the Geodetic Envelope and Polsby-Popper 
scores over all the districts of each state. The goal of this comparison is to measure the districting of 
states on average. The states with the lowest scores are
\begin{table}[H]
\centering
\begin{tabular}{|c|c|}
\hline
Geodetic Envelope & Polsby-Popper \\ \hline
\begin{tabular}{c|c|c|c}
Rank & State & Nr. Districts & Score \\
\hline
1 & AK & 1  & 0.0342 \\ \hline
2 & HI & 2  & 0.1566 \\ \hline
3 & RI & 2  & 0.3024 \\ \hline
4 & LA & 6  & 0.3269 \\ \hline
5 & MA & 9  & 0.3586 \\ \hline
6 & IL & 17 & 0.3652 \\ \hline
7 & NJ & 11 & 0.3669 \\ \hline
8 & ME & 2  & 0.3726 \\ \hline
9 & MD & 8  & 0.3770 \\ \hline
10 & SC & 7 & 0.3959 \\
\end{tabular}
&
\begin{tabular}{c|c|c|c}
Rank & State & Nr. Districts & Score \\
\hline
1 & AK & 1  & 0.0001 \\ \hline
2 & LA & 6  & 0.0849 \\ \hline
3 & HI & 2  & 0.1334 \\ \hline
4 & IL & 17 & 0.1452 \\ \hline
5 & NH & 2  & 0.1512 \\ \hline
6 & RI & 2  & 0.1645 \\ \hline
7 & MA & 9  & 0.1694 \\ \hline
8 & TX & 38 & 0.1804 \\ \hline
9 & ME & 2  & 0.1807 \\ \hline
10& CA & 52 & 0.1820 \\
\end{tabular}
\\ \hline
\end{tabular}
\caption{Bottom 10 States}
\end{table}
To interpret this ranking it is helpful to consider only those states with at least 3 districts excluding 
states AK, HI, NH and RI. According to both the Geodetic Envelope and Polsby-Popper score, the bottom 
three states are LA, MA and IL.

\section{Conclusion}
The Geodetic Envelope Score was shown to be a natural measure of district compactness because it is
defined using ``parallels'' and ``meridians''; the lines used in the original definition of state boundaries.
A number of methods were developed to compute this score stemming from a sundry of applied mathematical
disciplines. The characteristics and limitations of each method was summarized.
In comparison with the Polsby-Popper score the Geodetic Envelope Score computed values more in line
expectations, especially for reasonably shaped districts. 
\appendix
\renewcommand{\theequation}{\Alph{section}.\arabic{equation}}
\setcounter{equation}{0}

\newpage
\section{Appendix}
\subsection*{Formulas}
The elliptical integral of the second kind is given by
\[
E(e^2) = \int_0^{2\pi} \sqrt{1 - e^2sin^2(\theta)} \, d\theta
\]
The eccentricity of an ellipse, also called the first eccentricity is given by
\[
e^2 = \frac{a^2-b^2}{a^2}
\]
The second eccentricity is given by
\[
e'^2 = \frac{a^2-b^2}{b^2}
\]
\subsection*{Derivation of Surface Area of Envelope}
The surface area of the surface of revolution generated by revolving the graph of $x=x(y)$ about the y-axis is
\begin{equation}
    S = 2\pi\int_{y}^{b} x \sqrt{1 + (\frac{dx}{dy})^2} \,dy
\end{equation}
In this case, the surface of revolution is generated by revolving the `meridian' ellipse $x^2/a^2 + y^2/b^2 = 1$. Then
the surface area of the surface of revolution obtained from revolving a region from a given point $y$ up to $b$ is found 
by substituting $dx/dy = -a^2y/b^2x$ into the above equation
\begin{align*}
    S(y)&= 2\pi\int_{y}^{b}x\sqrt{\frac{b^4x^2+a^4y^2}{b^4x^2}} \,dy \\
        &= \frac{2\pi}{b^2}\int_{y}^{b}\sqrt{b^4x^2+a^4y^2} \, dy \\
        &= \frac{2\pi}{b^2}\int_{y}^{b}\sqrt{b^4(a^2-\frac{a^2}{b^2y^2})+a^4y^2} \, dy \\
        &= \frac{2\pi}{b^2}(ab)\int_{y}^{b}\sqrt{b^2-y^2+\frac{a^2}{b^2}y^2} \, dy \\
        &= 2\pi(\frac{a}{b})\int_{y}^{b}\sqrt{b^2+\frac{a^2-b^2}{b^2}y^2} \, dy \\
        &= 2\pi(\frac{a}{b})\int_{y}^{b}\sqrt{b^2+e'^2y^2} \, dy \\
\end{align*}
The surface area of the latitudinal band between points $y=y_1$ and $y=y_2$ is $S=S(y_1)-S(y_2)$, or
\begin{equation*}
    S = 2\pi\int_{y_1}^{y_2} \sqrt{b^2+e'^2y^2} \,dy
\end{equation*}
where $y_1 = N(1-e^2)sin(\varphi_1)$ and $y_1 = N(1-e^2)sin(\varphi_2)$. To evaluate the integral let $y=(b/e')tan(u)$. 
The integral becomes
\begin{equation*}
    S = 2\pi(\frac{ab}{e'})\int_{u_1}^{u_2} sec^3(u) \,du
\end{equation*}
The integral of $sec^3(x)$ can be found by integration by parts, letting $u=sec(x)$ and $dv=sec^2(x)dx$. The result is
\begin{equation*}
    S = (\frac{\pi}{e'})(ab)(sec(u)tan(u)+ln|sec(u)+tan(u)|)\Big|_{u_1}^{u_2}
\end{equation*}
If $\Delta s$ is the arc length corresponding to a change of longitude $\Delta\lambda$ in a circle of radius $R$ then
\begin{equation*}
    \frac{\Delta s}{2\pi R} = \frac{\Delta\lambda Ncos(\varphi)}{2\pi Ncos(\varphi)} = \frac{\Delta\lambda}{2\pi}
\end{equation*}
Multiplying this factor by the above equation and allowing for $\lambda$, $\varphi$ to be given in degrees, the surface area
of a geodetic envelope is
\begin{equation*}
    S = \frac{ab}{2e'}(sec(u)tan(u)+ln|sec(u)+tan(u)|)\Big|_{u_1}^{u_2}(\lambda_2-\lambda_1)(\frac{\pi}{180})
\end{equation*}
A similar derivation can be done for \eqref{eq:SASphere} by assuming the earth geoid is 
modeled by a sphere of radius $a$.
\newpage
\printbibliography
\end{document}
