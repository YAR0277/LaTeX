\documentclass{article}
\author{David Gore}
\title{The Geodetic Envelope Score}
\date{December 8, 2025}

\usepackage[
    backend=biber,
    style=numeric,
    sorting=none
    ]{biblatex}
\addbibresource{references.bib}

\usepackage{amsthm}
\usepackage{amsmath}
\usepackage{graphicx}
\usepackage{float}

% Numbered examples without section number
\newtheorem{example}{Example}

\begin{document}
\maketitle

\section{Introduction}
Several metrics have been developed to quantify the degree to which congressional districts have
been gerrymandered. Many of these metrics measure the compactness of a district by 
comparing a geometrical property of the district, such as area or perimeter, to a geometrical property 
in a related ideal shape. The ideal shapes are typically circles or rectangles. 
For example, the Reock score is the ratio of the area of a district to the area of the minimum 
circumscribing circle of the district. The Polsby-Popper score is the ratio of the area of a district 
to the area of a circle whose perimeter is the same as the district's perimeter. The Schwartzberg score
is the ratio of the perimeter of a district to the perimeter of a circle having the same area as the 
district.

\section{The Geodetic Envelope Score}
The book ``How The States Got Their Shapes'', \cite{stein}, documents the history of how the states got 
their individual boundaries. Boundaries can be natural, political, based on the division of resources, 
based on a principle such as equality, or be given in the King's charter. In all cases, the most common state 
delineators were defined by ``parallels'', lines of latitude, and ``meridians'', line of longitude. 
Kansas, Nebraska, South Dakota and North Dakota all have 3° of latitudinal height. While
Colorado, Wyoming and Montana all have 4°. The states Washington, Oregon, Colorado, Wyoming,
North Dakota and South Dakota all have about 7° of longitudinal width. The southern borders of Virginia,
Kentucky and Missouri are all at 37.5° latitude. Indeed a large section of the northern border of the 
United States is at 49° latitude. Thus, it seems natural to use lines of latitude and longitude in the 
definition of a compactness score for voting districts.

A \textit{geodetic envelope} or simply \textit{envelope} is a Geographic Information System (GIS) / Open Geospatial
Consortium (OGC) designated term defining a region specified by two latitude-longitude points. The two points are 
taken to be $(\lambda_{min},\varphi_{min})$ and $(\lambda_{max},\varphi_{max})$, where $\lambda$ denotes
longitude and $\varphi$ denotes geodetic latitude.

The geodetic envelope score of a district is the area of the district divided by the area of the geodetic
envelope containing the district,
\begin{equation}
    GE = \frac{A(D)}{A(S)}
\end{equation}
Here, $D$ is the district and $S$ is the bounding envelope of $D$. $S$ is a surface on the reference ellipsoid, 
a surface which approximates the earth geoid.
The surface area of $S$, $A(S)$, can be computed by rotating the ``meridian'' ellipse, 
${x^2}/{a^2} + {y^2}/{b^2} = 1$ with semi-major axis $a=6378.137$ km and semi-minor axis $b=6356.752$ km
about its minor axis. The values for $a$ and $b$ are defined by the world geodetic system (WGS)-84. 
Also, the first eccentricity of the earth, $e$ is related to the ellipses' axes by the formula
$e=((a^2-b^2)/a^2)^{1/2}$ and has the value $e=0.081819$, \cite{noureldin}. 
The area of the surface generated by rotating a horizontal
element of area from $y=y_1$ to $y=y_2$ about the y-axis is \cite{engmath}
\begin{equation}
    A(S) = 2\pi(\frac{a}{b})\int_{y_1}^{y_2}\sqrt{b^2+e'^2y^2} dy
\label{eq:SurfaceArea}
\end{equation}
$e'$ is called the second eccentricity of the earth. Its formula is $e'=((a^2-b^2)/b^2)^{1/2}$ and
it has the value $e'=0.082094$. 
If $a=b$, $e'=0$ and $y_1=0$, $y_2=a$, then \eqref{eq:SurfaceArea} evaluates to $A(S)=2\pi a^2$, the surface area 
of a hemi-sphere. Evaluating the integral in \eqref{eq:SurfaceArea} leads to the
final result for the surface area of an envelope where longitude ($\lambda$) and 
latitude ($\varphi$) are given in degrees is
\begin{equation}
    A(S) = \frac{ab}{2e'}(sec(u)tan(u)+ln|sec(u)+tan(u)|)\Big|_{u_1}^{u_2}(\lambda_2-\lambda_1)(\frac{\pi}{180})
\label{eq:SurfaceAreaEnvelope}
\end{equation}
The limits of integration are $u_1 = tan^{-1}((e'/b)y_1)$, $u_2 = tan^{-1}((e'/b)y_2)$ where 
$y_1 = N(\varphi_1)(1-e^2)sin(\varphi_1)$ and $y_2 = N(\varphi_2)(1-e^2)sin(\varphi_2)$. 
Details are shown in the Appendix.

\subsection{Spherical Approximation}
If, instead of the reference ellipsoid, a sphere of radius $a$ is taken as an approximation to the earth geoid,
\eqref{eq:SurfaceAreaEnvelope} simplifies to 
\begin{equation}
    A(S) = a^2(sin(\varphi_2)-sin(\varphi_1))(\lambda_2-\lambda_1)(\frac{\pi}{180})
\label{eq:SurfaceAreaEnvelopeSphere}
\end{equation}
The differences in calculated area of district envelopes between using \eqref{eq:SurfaceAreaEnvelope} and 
\eqref{eq:SurfaceAreaEnvelopeSphere} are shown in the following figure. In this figure the
districts have been sorted by the median latitude of the envelope. Hence, districts with lower
district numbers have a smaller median latitude than districts with higher district numbers.
\begin{figure}[H]
    \centering
    \includegraphics[width=0.7\textwidth]{cmpAreaSphere.png}
    \caption{Envelope Area Differences for Spherical Earth.}
    \label{fig:CmpAreaSphere}
\end{figure}

The minimum value of -588 $km^2$ at index $23$ is district TX23. This Texas district is on the 
United States' southern border with Mexico.
The maximum value of 657 $km^2$ at index $415$ is district MT02. This Montana district is on the 
United States' northern border with Canada. In other words, the spherical approximation overestimates
the envelope area for states in the southern United States and underestimates envelope area for states 
in the northern United States.

The relative error in percent of the spherical earth model approximation ranges from a minimum value
of 0.0009 at latitude 37.785\textdegree to maximum values 0.2634, 0.2454 at latitudes 25.101\textdegree, 
48.390\textdegree, respectively.

\subsection{Flatland Approximation}
Another possibility is to project the district envelope onto two dimensional Euclidean space and compute
the area there. The Equal Earth Projection \cite{ee2018} accomplishes this task. 
The differences in calculated area of district envelopes between using \eqref{eq:SurfaceAreaEnvelope} and
calculating the area in Euclidean space are shown in the following figure. In this figure
districts have been sorted by the area of the district. Lower district numbers have smaller areas than
districts with higher district numbers.
\begin{figure}[H]
    \centering
    \includegraphics[width=0.7\textwidth]{cmpAreaFlatland.png}
    \caption{Envelope Area Differences using Equal Earth Projection.}
    \label{fig:CmpAreaFlatland}
\end{figure}

The minimum value of -5052 $km^2$ at index $403$ is district MI01. This Michigan district is a large 
district whose envelope has an area of 314,056 $km^2$.
The maximum value of -0.0185 $km^2$ at index $3$ is district NY09. This New York district is a small district 
whose envelope's area is 70 $km^2$. This illustrates the shape distortion in an equal-area projection:
the greater the area being mapped, the greater the distortion becomes.

The relative error in percent of the flatland approximation ranges from a minimum value
of 0.0235 to a maximum value of 2.4502.

\section{Comparison with Polsby-Popper Score}
As representative to compare with the Polsby-Popper score was chosen. 
The Polsby-Popper (PP) score \cite{pp1991} is the ratio of the area of a district to the area of a circle 
whose circumference is equal to the perimeter of the district. If $A$ is the area operator and $P$ 
the perimeter operator then the PP-score is

\begin{equation}
PP = \frac{A(D)}{A(R)}
\label{eq:PP1}
\end{equation}
Here, $D$ represents a district and $R$ a circle with $P(D) = P(R)$. If $d$ is the diameter of circle $R$ then 
$A(R)=\pi(d^2/2)$ and substituting into \eqref{eq:PP1},

\begin{equation}
PP = 4\pi\frac{A(D)}{P(D)^2}
\label{eq:PP2}
\end{equation}
Two applications of \eqref{eq:PP2} are shown in the following examples.
\begin{example}
Suppose $D$ is a rectangle with height $h$ and length $l$. The Polsby-Popper score for the rectangle is
\[
PP = \pi\frac{lh}{(l+h)^2}
\]
If $l=h$ (a square) then $PP=0.79$, if $l=2$, $h=1$ then $PP=0.70$, and if $l=4$, $h=1$ then $PP=0.50$.
\end{example}

\begin{example}
Suppose $D$ is an ellipse with semi-major axis $a$ and semi-minor axis $b$. Let $C$ denote the boundary of
the ellipse and choose the parametric equations for $C$ to be $x=acos(t)$, $y=bsin(t)$ for $0 \le t \le 2\pi$. 
Thus $C$ is traversed in the counter-clockwise (positive) direction. Then using Green's Theorem in the 
plane \cite{engmath}, with $P=-y/2$ and $Q=x/2$,
\[
A(D) = \oint_C -\frac{y}{2} \, dx + \frac{x}{2} \, dy = \int_0^{2\pi} [\frac{ab}{2}sin^2(t)+\frac{ab}{2}cos^2(t)] \, dt = \pi{ab}
\]
Furthermore, the perimeter of the ellipse can be written \cite{abramowitz}
\[
P(D) = 4aE(e)
\]
where $E$ is the elliptical integral of the second kind and $e$ is the eccentricity of the ellipse. The
formulas for $E$ and $e$ are shown in the Appendix. The Polsby-Popper score for the ellipse is
\[
PP = \frac{\pi}{4}\frac{b}{a}[E^2(e)]^{-1}
\]
If $a=b$ (a circle) then $PP=1$, if $a=2$, $b=1$ then $PP=0.71$, and if $a=4$, $b=1$ then $PP=0.49$.
\end{example}
These examples show that comparing the area of a district $D$ to the area of a circle $R$ as 
in \eqref{eq:PP1} can lead to low compactness scores even for reasonably shaped districts 
such as rectangles or ellipses. 
This fact, illustrated for the Polsby-Popper score, would also hold true for the other metrics
based on the circle: Roeck and Schwartzberg.
This is obvious since the circle doesn't cover these reasonably shaped districts very well.

\section{Results}
The following histogram compares the Geodetic Envelope with the Polsby-Popper score for all districts. 
The average value of the Geodetic Envelope score, $0.4724$, is almost twice that of the 
Polsby-Popper score, $0.2465$. This implies that geodetic envelopes are more suited as a compactness 
measure for districts than circles.
\begin{figure}[H]
    \centering
    \includegraphics[width=0.7\textwidth]{hist.png}
    \caption{Histogram of Geodetic Envelope and Polsby-Popper score.}
    \label{fig:Histogram}
\end{figure}

The following table compares the districts with the top 5 Geodetic Envelope and Polsby-Popper scores.
\begin{table}[H]
\centering
\begin{tabular}{|c|c|}
\hline
Geodetic Envelope & Polsby-Popper \\ \hline
\begin{tabular}{c|c|c}
Rank & District & Score \\
\hline
1 & WY00 & 0.9885 \\ \hline
2 & ND00 & 0.9281 \\ \hline
3 & IN07 & 0.9097 \\ \hline
4 & MT02 & 0.8826 \\ \hline
5 & NV02 & 0.8629 \\
\end{tabular}
&
\begin{tabular}{c|c|c}
Rank & District & Score \\
\hline
1 & WY00 & 0.7596 \\ \hline
2 & IN07 & 0.6953 \\ \hline
3 & NV02 & 0.5769 \\ \hline
4 & FL15 & 0.5646 \\ \hline
5 & CO05 & 0.5573 \\
\end{tabular}
\\ \hline
\end{tabular}
\caption{Top 5 Districts}
\label{tab:top5-districts}
\end{table}
District ND00 encompasses the entire state of North Dakota and its shape is a reasonable looking
rectangle. Its Geodetic Envelope score of $0.9281$ ranks second among all districts. 
If the Polsby-Popper score is used, ND00 ranks 20th with a score of $0.4761$. This further supports
the notion that the Geodetic Envelope is a more suitable measure of compactness than Polsby-Popper.

The districts AK00 and HI02 in Alaska and Hawaii, respectively, have by far the worst compactness scores.
For example, the Polsby-Popper score of AK00 is $O(10^{-4})$. Thus, when comparing the bottom 5 districts,
these states were considered outliers and removed from the analysis. The next table compares the districts 
with the lowest Geodetic Envelope and Polsby-Popper scores 
\begin{table}[H]
\centering
\begin{tabular}{|c|c|}
\hline
Geodetic Envelope & Polsby-Popper \\ \hline
\begin{tabular}{c|c|c}
Rank & District & Score \\
\hline
1 & CA11 & 0.0424 \\ \hline
2 & CA42 & 0.0677 \\ \hline
3 & TX35 & 0.1137 \\ \hline
4 & FL28 & 0.1192 \\ \hline
5 & OH11 & 0.1399 \\
\end{tabular}
&
\begin{tabular}{c|c|c}
Rank & District & Score \\
\hline
1 & CA42 & 0.0260 \\ \hline
2 & CA11 & 0.0295 \\ \hline
3 & TX33 & 0.0366 \\ \hline
4 & LA06 & 0.0505 \\ \hline
5 & CA41 & 0.0596 \\
\end{tabular}
\\ \hline
\end{tabular}
\caption{Bottom 5 Districts}
\label{tab:bottom5-districts}
\end{table}

Next, a state-wide comparison is made by calculating the average of the Geodetic Envelope and Polsby-Popper 
scores over all the districts of each state. The goal of this comparison is to measure the districting of 
states on average. The states with the lowest scores are
\begin{table}[H]
\centering
\begin{tabular}{|c|c|}
\hline
Geodetic Envelope & Polsby-Popper \\ \hline
\begin{tabular}{c|c|c|c}
Rank & State & Nr. Districts & Score \\
\hline
1 & AK & 1  & 0.0342 \\ \hline
2 & HI & 2  & 0.1566 \\ \hline
3 & RI & 2  & 0.3024 \\ \hline
4 & LA & 6  & 0.3269 \\ \hline
5 & MA & 9  & 0.3586 \\ \hline
6 & IL & 17 & 0.3652 \\ \hline
7 & NJ & 11 & 0.3669 \\ \hline
8 & ME & 2  & 0.3726 \\ \hline
9 & MD & 8  & 0.3770 \\ \hline
10 & SC & 7 & 0.3959 \\
\end{tabular}
&
\begin{tabular}{c|c|c|c}
Rank & State & Nr. Districts & Score \\
\hline
1 & AK & 1  & 0.0001 \\ \hline
2 & LA & 6  & 0.0849 \\ \hline
3 & HI & 2  & 0.1334 \\ \hline
4 & IL & 17 & 0.1452 \\ \hline
5 & NH & 2  & 0.1512 \\ \hline
6 & RI & 2  & 0.1645 \\ \hline
7 & MA & 9  & 0.1694 \\ \hline
8 & TX & 38 & 0.1804 \\ \hline
9 & ME & 2  & 0.1807 \\ \hline
10& CA & 52 & 0.1820 \\
\end{tabular}
\\ \hline
\end{tabular}
\caption{Bottom 5 States}
\label{tab:bottom5-states}
\end{table}
To interpret this ranking it is helpful to consider only those states with at least 3 districts. Excluding 
states with a small number of districts and low compactness scores (AK, HI, NH and RI) one can see that
LA, MA and IL have the lowest compactness scores by either measure.

\section{Conclusion}
It was shown that the Geodetic Envelope Score is a natural measure of district compactness because it is
defined using ``parallels'' and ``meridians''; the lines used in the original definition of state boundaries.
A comparison to the Polsby-Popper score showed that the Geodetic Envelope Score gives more consistent values
for reasonable district shapes. Two approximations to the Geodetic Envelope Score were given: one based on
the spherical earth approximation and one based on projecting envelopes to Euclidean space.
The limitations of each approximation were illustrated.
\appendix
\renewcommand{\theequation}{\Alph{section}.\arabic{equation}}
\setcounter{equation}{0}

\newpage
\section{Appendix}
\subsection*{Formulas}
The elliptical integral of the second kind is given by
\[
E(e^2) = \int_0^{2\pi} \sqrt{1 - e^2sin^2(\theta)} \, d\theta
\]
The eccentricity of an ellipse, also called the first eccentricity is given by
\[
e^2 = \frac{a^2-b^2}{a^2}
\]
The second eccentricity is given by
\[
e'^2 = \frac{a^2-b^2}{b^2}
\]
\subsection*{Derivation of Surface Area of Envelope}
The surface area of the surface of revolution generated by revolving the graph of $x=x(y)$ about the y-axis is
\begin{equation}
    S = 2\pi\int_{y}^{b} x \sqrt{1 + (\frac{dx}{dy})^2} \,dy
\end{equation}
In this case, the surface of revolution is generated by revolving the `meridian' ellipse $x^2/a^2 + y^2/b^2 = 1$. Then
the surface area of the surface of revolution obtained from revolving a region from a given point $y$ up to $b$ is found 
by substituting $dx/dy = -a^2y/b^2x$ into the above equation
\begin{align*}
    S(y)&= 2\pi\int_{y}^{b}x\sqrt{\frac{b^4x^2+a^4y^2}{b^4x^2}} \,dy \\
        &= \frac{2\pi}{b^2}\int_{y}^{b}\sqrt{b^4x^2+a^4y^2} \, dy \\
        &= \frac{2\pi}{b^2}\int_{y}^{b}\sqrt{b^4(a^2-\frac{a^2}{b^2y^2})+a^4y^2} \, dy \\
        &= \frac{2\pi}{b^2}(ab)\int_{y}^{b}\sqrt{b^2-y^2+\frac{a^2}{b^2}y^2} \, dy \\
        &= 2\pi(\frac{a}{b})\int_{y}^{b}\sqrt{b^2+\frac{a^2-b^2}{b^2}y^2} \, dy \\
        &= 2\pi(\frac{a}{b})\int_{y}^{b}\sqrt{b^2+e'^2y^2} \, dy \\
\end{align*}
The surface area of the latitudinal band between points $y=y_1$ and $y=y_2$ is $S=S(y_1)-S(y_2)$, or
\begin{equation*}
    S = 2\pi\int_{y_1}^{y_2} \sqrt{b^2+e'^2y^2} \,dy
\end{equation*}
where $y_1 = N(1-e^2)sin(\varphi_1)$ and $y_1 = N(1-e^2)sin(\varphi_2)$. To evaluate the integral let $y=(b/e')tan(u)$. 
The integral becomes
\begin{equation*}
    S = 2\pi(\frac{ab}{e'})\int_{u_1}^{u_2} sec^3(u) \,du
\end{equation*}
The integral of $sec^3(x)$ can be found by integration by parts, letting $u=sec(x)$ and $dv=sec^2(x)dx$. The result is
\begin{equation*}
    S = (\frac{\pi}{e'})(ab)(sec(u)tan(u)+ln|sec(u)+tan(u)|)\Big|_{u_1}^{u_2}
\end{equation*}
If $\Delta s$ is the arc length corresponding to a change of longitude $\Delta\lambda$ in a circle of radius $R$ then
\begin{equation*}
    \frac{\Delta s}{2\pi R} = \frac{\Delta\lambda Ncos(\varphi)}{2\pi Ncos(\varphi)} = \frac{\Delta\lambda}{2\pi}
\end{equation*}
Multiplying this factor by the above equation and allowing for $\lambda$, $\varphi$ to be given in degrees, the surface area
of a geodetic envelope is
\begin{equation*}
    S = \frac{ab}{2e'}(sec(u)tan(u)+ln|sec(u)+tan(u)|)\Big|_{u_1}^{u_2}(\lambda_2-\lambda_1)(\frac{\pi}{180})
\end{equation*}
A similar derivation can be done for \eqref{eq:SurfaceAreaEnvelopeSphere} by assuming the earth geoid is 
modeled by a sphere of radius $a$.
\newpage
\printbibliography
\end{document}
